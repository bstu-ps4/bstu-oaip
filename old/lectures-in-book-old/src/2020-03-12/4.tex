\subsection{Классы памяти и области действия объектов. Динамическая и статическая память}


Одним из атрибутов в декларации объектов является класс памяти. Он определяет время существования переменной (время жизни) и область видимости.

Есть 3 места, где объявляется переменные:
\begin{enumerate}
    \item внутри функции
    \item при определении параметра функции
    \item вне функции
\end{enumerate}

Эти переменные соотвественно называются:
\begin{enumerate}
    \item локальные
    \item формальные параметры
    \item глобальные переменные
\end{enumerate}

\begin{itemize}
    \item Динамическая память, которая выделяется при вызове функции и освобождается при выходе из неё
    \begin{itemize}
        \item auto - автоматический
        \item regiter - регистровый
    \end{itemize}

    \item Статическая память, которая распределяет на этапе трансляции и заполняется по умолчанию нулями
    \begin{itemize}
        \item extern - внешний
        \item static - статический
    \end{itemize}
\end{itemize}

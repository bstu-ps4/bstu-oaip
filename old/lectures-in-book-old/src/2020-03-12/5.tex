\subsection{Автоматические переменные}

Переменные декларируемые внутри функции являются внутреними (или локальными) и никакая другая функция не имеет прямого доступа к ним. Такие объекты существуют времено на этапеактивности функции.

Каждая локальная переменная существует только в блоке кода, в котором объявлена, и уничтожается при выходе из него. Эти переменные называют \underline{автоматическими}.

Атрибут класса памяти \texttt{auto} локальные объекты получаю поумолчанию, хотя и принадлежность к этому классу можно указать явно.

\underline{Класс памяти в области действия объекта}

\begin{verbatim}
void main() {
    auto int max, lin;
    ...
}

int sp;
double val[20];

extern int sp;
extern double val[];
\end{verbatim}

Так получают, если... , что определение переменной не нужно искать вне функции.
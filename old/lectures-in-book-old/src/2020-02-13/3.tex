\subsection{Присваивание структур}

Информация, содержащаяся в одной структуре может быть присвоена другой структуре того же типа с помощью одиночного оператора присваивания. То есть не нужно присваивать значение каждого поля по отдельности.

Программа демонстрирует присваивание структур:

\begin{verbatim}
#include <stdio.h> 

int main (void) { 
    struct { 
        int a;
        int b;
    } x, y;

    x.a = 10;
    x.b = 20;
    y = x; /* присвоение одной структуры другой */
    printf("Contents of y: %d %d.", y.a, y.b);

    return 0;
}
\end{verbatim}

После присваивания переменные \texttt{y.a} и \texttt{y.b} будут содержать значения \texttt{10} и \texttt{20} соответственно.

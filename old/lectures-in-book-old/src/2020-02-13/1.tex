\subsection{Структуры. Объединения. Определяемые пользователем типы}

Язык Си предоставляет 5 способов создания своих типов данных:

\begin{enumerate}
\item Структура
    – совокупность переменных, объединенных одним именем.

\item Битовое поле
    – разновидность структуры, предоставляющая легкий доступ к отдельным битам

\item Объединение
    – позволяет одному участку памяти содержать два или более различных типов данных.

\item Перечисление
    – список символов.

\item Typedef
    – ключевое слово, которое создает новое имя существующему типу.

\end{enumerate}

\underline{Структура} – совокупность переменных, объединенных одним именем, представляющая общепринятый способ совместного хранения информации. 

Объявление структуры приводит к образованию шаблона, который дальше используется для создания объектов структуры.


\underline{Переменные, образующие структуры}, называются \underline{полями} или \underline{элементами структуры}.


Обычно все поля структуры связаны друг с другом (по смыслу).

Объявление шаблона, определяющего имя, адрес: 


Ключевое слово struct объявляет компилятору об объявлении структуры. 

\begin{verbatim}
struct addr { 
    char name [30];
    char street [40];
    char city [20];
    char state [3];
    unsigned long int zip;
};
\end{verbatim}

Объявление переменной:

\begin{verbatim}
struct addr addr_info;
\end{verbatim}

Объявление структуры заканчивается точкой с запятой (\texttt{;}), потому что это оператор. Имя структуры (в нашем случае \texttt{addr}) идентифицирует структуру и объявляется спецификатором типа.

На данный момент ещё не создана переменная, создан только тип и определена форма данных. Для определения переменной созданной структуры нужно написать: 

\begin{verbatim}
struct addr addr_info; 
\end{verbatim}


В этой строке происходит объявление \texttt{addr\_info} типа \texttt{addr}.

Когда объявлена структурная переменная, компилятор автоматически выделяет необходимый участок памяти для размещения всех полей.

Размещение структуры \texttt{addr\_info} в памяти:

\begin{verbatim}
              ___
Name   30bytes  |
Street 40 bytes |
City   20 bytes |__ addr_info
State  3 bytes  |
Zip    4 bytes  |
             ___|
\end{verbatim}

Стандартный вид объявления структуры: 

\begin{verbatim}
struct ярлык {
    тип имя переменной;
    тип имя переменной;
    тип имя переменной;
} структурые переменные;
\end{verbatim}

\underline{Ярлык} – имя типа структуры, а не имя переменной.

\underline{Структурные переменные} – разделенный запятыми список имён переменных.

Следует помнить, что или ярлык или структурные переменные могут отсутствовать, но не оба.

Можно объявлять более естественную структуру переменных: 

Объявление 1-ой переменной: 

\begin{verbatim}
struct { 
    char name [30];
    сhar street [40];
    char city [20];
    char state [3];
    unsigned long int zip;
} addr_ info;
\end{verbatim}

Объявление нескольких переменных:

\begin{verbatim}
struct addr { 
    char name [30];
    сhar street [40];
    char city [20];
    char state [3];
    unsigned long int zip;
} addr_info, binfo, cinfo;
\end{verbatim}

Каждая новая структурная переменная содержит свою собственую копии... Между ними нет связи, они просто являются экземплярами одного типа структуры.
\subsection{Доступ к полям структуры}

Доступ к отдельным полям структуры осущeствляется с помощью оператора точка (\texttt{.}).

Стандартный вид доступа:

\begin{verbatim}
имя_структуры.имя_поля
\end{verbatim}

Примеры: 
\begin{itemize}
\item присваиваем полю \texttt{zip} структурной переменной \texttt{addr\_info} значение \texttt{12345}: 
    \begin{verbatim}
addr_info.zip = 12345;
    \end{verbatim}

\item выводим поле \texttt{zip} на экран: 
    \begin{verbatim}
printf ("%ld", addr_info.zip);
    \end{verbatim}

    Эта строка выводит на экран структурное поле \texttt{zip}.

\item передаем указатель на символ, указывающий на начало name: 
    \begin{verbatim}
gets (addr_info.name);
    \end{verbatim}
  
    Использует массив символом \texttt{addr.name}.
  
    Эта команда передаёт символ указывающий на начало \texttt{name}.
\end{itemize}
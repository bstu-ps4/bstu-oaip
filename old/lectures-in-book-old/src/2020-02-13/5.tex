\subsection{Передача структур в функции}

Рассмотрим передачу полей и структур в функции. 

В этом случае фактически передается значение поля, то есть обычная переменная.

В примере объявляем структуру и затем передаем каждое поле этой структуры в функцию:

\begin{verbatim}
struct frend { 
    char x;
    int y;
    float z;
    char s[10];
} mike;
\end{verbatim}

Примеры передачи каждого поля в функцию:

\begin{verbatim}
func(mike.x); /* передача символьного значения x */
func2(mike.y); /* передача целочисленного значения y */
func3(mike.z); /* передача вещественного значения z */
func4(mike.s); /* передача адреса строки s */
func(mike.s[2]); /* передача символьного значения s[2] */ 
\end{verbatim}

Если необходимо передать адрес отдельного поля структуры, следует поместить \texttt{\&} перед именем структуры: 

\begin{verbatim}
func(&mike.x); /* передача адреса символа x */
func2(&mike.y); /* передача адреса целого y */
func3(&mike.z); /* передача адреса вещественного z */
func4(mike.s); /* передача адреса строки s */
func(&mike.s[2]); /* передача адреса символа s[2] */
\end{verbatim}

Нужно обратить внимание, что \texttt{\&} cтоит перед именем структуры, а не перед именем поля. Массив \texttt{s} сам по себе является адресом, поэтому не надо адреса, но когда осуществляется доступ к символу строки \texttt{s} (5 строка), то \texttt{\&} необходим.

Передача всей структуры в функцию.

Когда структуры используются как аргумент функции, передается вся структура с помощью стандартной передачи по значению. (копия значения, то есть изменения не влияют на структуру, используемую в качестве аргумента).

Так же важно помнить, что тип аргумента должен соответствовать типу параметра. Для этого лучше всего определить структуру глобально и затем использовать её ярлык для объявления неоходимых структурных переменных и параметров. 

\begin{verbatim}
#include <stdio.h>

/*объявление типа структуры*/
struct struct_type { 
    int a, b;
    char ch;
};

void f1 (struct_type parm);

int main (void) { 
    struct struct_type.arg; /* объявление arg */
    arg.a = 1000;
    f1(arg);

    return 0;
}

void f1(struct struct_type parm) { 
    printf ("%d", parm.a);
}
\end{verbatim}

Данная программа выводит число \texttt{1000} на экран.

Можно видеть \texttt{arg} и \texttt{param} объявлены типа \texttt{struct\_type}.

\subsection{Битовые поля}

В Си есть возможность, которая называется битовыми полями, что позволяет работать с отдельными видами.

Причины по которым полезны битовые поля:
\begin{enumerate}
\item Можно сохранить несколько переменных в 1-ом байте
\item некоторые интэрфэйсы устройств передают информацию закодировав бит в 1-ин байт
\item существует процедуры кодировани, которые осуществляют доступ к отдельному биту
\end{enumerate}

Метод использования битовых полей основан на структурах

\underline{Битовое поле} - особый вид структуры определяющий какую длину имеет кодовое поле.

Стандартный вид оъявления битовых полей:

\begin{verbatim}
struct имя_структуры {
    тип имя1: длина;
    тип имя2: длина;
    ...
    тип имяN: длина;
}
\end{verbatim}

тип - int, unsigned, signed.

Битовые поля должны объявляться и иметь длину 1-16 бит (в 16 битных средах)

Пример:

\begin{verbatim}
struct device {
    unsigned active: 1;
    unsigned ready: 1;
    unsigned error: 1;
} dev_code;
\end{verbatim}

Данная структура определяет тип переменной по 1-ому биту каждому.

Структурная переменная \texttt{dev\_code} может использована для декодирования информации порта ленточного накопителя.

Для такого гипотетического ленточного накопителя следующий фрагмент кода записывает байт информации на ленту и проверяет на ошибки, используя \texttt{dev\_code}:

\begin{verbatim}
void wr_tape(char c) {
    while(!dev_code.ready)
        rd(&dev_code); //ждать
    wr_to_tape(c); //запись байта
    while(dev_code.active)
        rd(&dev_code); //ожидание кончания записи информации
    if(dev_code.error)
        printf("Write Error");
}
\end{verbatim}

Здесь \texttt{rd()} возвращает статус ленточного носителя, \texttt{wr\_to\_tape()} записывает данные.

Рисунок показывает как выглядит переменная \texttt{dev\_code} в памяти:

...картинка...

К полю происходит обращение с помощью оператора "точка". Если обращение к структуре происходит с помощью указателя, то вместо "точки", используется оператор "стрелка".

Нет необходимости обзывать каждое битовое поле.

Если ленточный накопитель возвращает информацию, о наступлении конца ленты в 5-ом байте, следует применить структуру \texttt{device} следующим образом:

\begin{verbatim}
struct device {
    unsigned active: 1;
    unsigned ready: 1;
    unsigned error: 1;
    unsigned : 2;
    unsigned EOT: 1;
} dev_code;
\end{verbatim}

Ограничение битовых полей:

- нельзя получить адрес переменной битового поля
- переменные битового поля не могут помещаться в массив
- переходя с компьютера на компьютер нельзя быть уверены... зависит от компьютера

Можно смещивать различные структурные переменные в битовых полях:

\begin{verbatim}
struct emp {
    struct addr address;
    float pay;
    unsigned lay_off: 1;
    unsigned hourly: 1;
    unsigned deductions: 3;
};
\end{verbatim}

определяет запись служащего, использующую только 1-ин байт для хранения 3-ех частей информации статуса служащего...
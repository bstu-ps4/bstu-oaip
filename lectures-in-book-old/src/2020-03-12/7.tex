\subsection{Класс памяти static}

Объекта размещаемые ... с класом \texttt{static} и могут иметь любой атрибут области действия.

Глобальные объекты всегда являются статическими.

Атрибут \texttt{static} использованный при описании глобального объекта предписыввает ограничение области пременимости только в пределах остатка текущего файла.

Значение локальных статических объектов сохраняется при повторном вызове функции.

Таким образом в Си ключевое слово \texttt{static} имеет разный смысл для локальных и глобальных объектов.

Так как глобальные переменные доступны по всюду их можно использовать вместо списка аргументов для передачи значения между функциями, хотя это считается дурным тоном.

Внешние переменные существует постоянно сохраняют свои значения и после того как функция привсвоившее эти значения завершает функцию.

При отсутствии явной инициализации для внешних и статических переменных гарантируется их обнуление.

Автоматические регистровые имеют неопределённые начальные значения (мусор).

Внешние переменные должны быть определена вне всей функции. При этом этой ей выделяется фактическое место в памяти. Такая переменная должна быть описана в каждом функции, где она буудет использована. Это можно сделать явно, либо описание \texttt{extern}, либо по контексту.

Чтобу функция могла использовать внешнию переменную ей нужно сообщить..., то есть включить в функцию описание \texttt{extern}. Иногда описание \texttt{extern} может быть опущено. Если внешнее определение переменной находится в том же файле рание её использование в некоторой функции. В этом случае не обязательно включать описание \texttt{extern} для этой переменной в саму функцию.

 Обычная практика заключается в помещение ... всех переме... в начале файла. Затем опускают написание \texttt{extern}.

 Включение ключевого слова \texttt{extern} позволяет функции использовать внешнию переменную, даже если она определяется позже в этом или другом файле.

 Важно различать описание внешней переменной и её определение описание указывают размер, тип и так далее. Определение вызывает ещё отведение памяти.
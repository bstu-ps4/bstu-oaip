\subsection{Использование перечислений типа для обращения к полям объекта}

\begin{verbatim}
int main() {
    enum paytype{CARD, CHECK}; /* две формы оплаты с числовыми значениями 0, 1 */
    paytype ptype; /* определение переменной перечисления типа */
    union {
        char card[25];
        long check;
    } info;
\end{verbatim}

Эту форма характеризует плату наличкой или карточкой.

Возможно присваение значения объединению:

\begin{verbatim}
    p type = CARD; /* переменная флаг получает значение CARD(0) */
    strcpy(info.card, "12345"); /* объединение получает своё значение через none
card или */
    p type = pay type(1); /* переменная флаг получает значение CHECH(1) */
    info.check = 105l; /* объединение получает своЁ значение через поле check */
\end{verbatim}

Вывод значения переменной:

\begin{verbatim}
    switch(ptype) {
        case CARD:
            cout << "card: " << info.card << endl;
            break;
        case CHECH:
            cout << "check: " << info.check << endl;
            break;
    }
    _getch();
    return 0;
}
\end{verbatim}
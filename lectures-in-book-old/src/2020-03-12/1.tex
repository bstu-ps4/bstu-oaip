\subsection{Перечисления}

\underline{Перечисления} - набор именованных целочисленных констант, определяющий все допустимые значения, которые может принимать переменная.

Перечисления определяются с помощбю ключевого слова \texttt{enum}, которое указывает на начало перечисления типа.

Перечисления можно встретить в повседневной жизни в качестве монет (центы США).

\begin{verbatim}
enum ярлык{список перечислений} список переменных;
\end{verbatim}

Ярлык или список переменных не обязателен, н он ...

Список перечислений разделеный запятыми...

Ярлык используется для переменных данного типа.

Следующий фрагмент определяет перечисления \texttt{coin} и объявляет переменную \texttt{money} этого типа:

\begin{verbatim}
enum coin{penny, nickel, dime, quarter, half_dollar, dollar};
enum coin money;
\end{verbatim}

Имея данное определение о объявление следующий тип присваивание совершенно корректен:

\begin{verbatim}
money = dime;
if (money == quarter) printf("is a quarter\n");
\end{verbatim}

В перечислениях каждому символу ставится в соотвествие целочисленное значение. И их можно использовать в любых целочисленных выражениях.

Например:

\begin{verbatim}
printf("The value of quanter is %d", quanter);
\end{verbatim}

Если явно не проводить инициализацию, значение первого символа перечисления будет 0, второго 1 и так далее.

Следовательно:

\begin{verbatim}
printf("%d %d", penny, dime); //0 2
//выводит 0 2 на экран
\end{verbatim}

Можно определить значение одного или несколько символов используя инициализатор. это...

При использовании инициализатора, символы следующие за инициализированным значением получают значение больше чем указанное перед этим.

Например:

В следующем объявлении \texttt{quanter} получает значение \texttt{100}:

\begin{verbatim}
enum coin (penny, nickel, dime, quanter = 100, half_dollar, dollar);
\end{verbatim}

Теперь символы получают следующие значения:

\begin{verbatim}
penny 0
nickel 1
dime 2
quanter 100
half_dollar 101
dollar 102
\end{verbatim}

Заблуждением считается возможность прямого ввода или вывода символа перечислений.

Слудеющий фрагмент кода не работает как нужно:

\begin {verbatim}
/* не работает */
money = dollar;
printf("%s", money);
\end{verbatim}

\texttt{dollar} - это имя целого числа, следовательно, не возможно с помощью \texttt{ptintf()} вывести строку \texttt{dollar}.

Надо помнить, что символ \texttt{dollar} - это просто имя целого числа, а не строка, следовательно, не возможно с помощью \texttt{printf()} вывести строку \texttt{dollar}, используя значение \texttt{money}.

Аналогично нельзя сообщить значение переменной перечисления, используя строковый эквивалент.

Таким образом, следующий код не работает:

\begin{verbatim}
/* этот код не будет работать */
money = "penny";
\end{verbatim}

Следующий код необходим для вывода состояния \texttt{money} с помощью слов:

\begin{verbatim}
switch(money) {
    case penny:
        printf("penny");
        break;
    case dime:
        printf("dime");
        break;
    case quarter:
        printf("quarter");
        break;
    case half_dollar:
        printf("half_dollar");
        break;
    case dollar:
        printf("dollar");
}
\end{verbatim}
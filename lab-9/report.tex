\documentclass[12pt,a4paper]{article}

% Данные для титульной страницы

% Номер лабораторной работы
\newcommand{\labnumber}{9}

% Название работы
\newcommand{\labtitle}{Строки в языке СИ}

% Группа студента
\newcommand{\group}{ПО-4}

% Студент, выполнивший работу
\newcommand{\labauthor}{Галанин П. И.}

% Должность преподавателя
\newcommand{\teacherstatus}{ст. преподаватель}

% Преподаватель
\newcommand{\teacher}{Хацкевич М. В.}

% Дата
\newcommand{\labdate}{2020}

% Подключение стилевого файла
\usepackage{../labwork} 

% Используемые языки программирования в исходниках в отчете
\lstloadlanguages{[ISO]C++}



\begin{document}



% Добавляем титульный лист
\maketitle



% Заголовок
\labheading



% Цель работы
\begin{labgoal}
Изученить принципы обработки строк в языке Си: ввод-вывод строк, использование стандартных функций языка С, работа с памятью.
\end{labgoal}



% Метка начала отчета по работе
\labreport




% Вывод
\begin{labconclusion}
Попробовали на практике строки: ввод (gets(str)), вывод (printf("\%s", str)), работь с динамической памятью (calloc(n, sizeof(char)), free(str)), использовать стандартные функции string.h (strtok()).
\end{labconclusion}



\end{document}

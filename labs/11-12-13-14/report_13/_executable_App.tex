%<<<===
\subsection{Индексный файл при открытии программы}

По коду мой индексный файл - это "indices.txt".

При открытии программки создается файл indices.txt, если он есть, то перезаписывается. Сейчас файл пустой:

\begin{tcolorbox}
\begin{verbatim}

\end{verbatim}
\end{tcolorbox}

В меню выбираю пункт 3, для вывода данных

\begin{tcolorbox}
\begin{verbatim}
Меню:
1. Открыть файл
2. Ввод данных       
3. Вывод данных      
4. Сортирова по полю 
5. Удалить запись    
6. Сохранить как     
0. Выйти из программы
\end{verbatim}
\end{tcolorbox}

Все выводится. А ведь используются индексный файл.

\begin{tcolorbox}
\begin{verbatim}
| ID   | байт Номер    | байт Марка    | байт Фамилия      | Осмотр     | 
| ---- | ------------- | ------------- | ----------------- | ---------- | 
Нажмите любую клавишу для продолжения...
\end{verbatim}
\end{tcolorbox}

Нажимаю любую клавишу для продолжения поподаю в меню.

\begin{tcolorbox}
\begin{verbatim}
Меню:
1. Открыть файл
2. Ввод данных       
3. Вывод данных      
4. Сортирова по полю 
5. Удалить запись    
6. Сохранить как     
0. Выйти из программы
\end{verbatim}
\end{tcolorbox}

Выбираю пункт 1, для открытия файла.

\begin{tcolorbox}
\begin{verbatim}
Размер пути файла: 
\end{verbatim}
\end{tcolorbox}

Ввожу размер пути. Он нужен для того, чтобы создать динамическую строку, а не статическую.

\begin{tcolorbox}
\begin{verbatim}
Размер пути файла: 10 
Какой файл открыть: 
\end{verbatim}
\end{tcolorbox}

Как открылся файл, то есть записал данные из файла в массив структур, то создался indices.txt, который содержит индексы упорядоченные по возврастанию.

\begin{tcolorbox}
\begin{verbatim}
0
1
2
3
\end{verbatim}
\end{tcolorbox}

В меню выбираю пункт 3, для вывода таблицы

\begin{tcolorbox}
\begin{verbatim}
Меню:
1. Открыть файл
2. Ввод данных       
3. Вывод данных      
4. Сортирова по полю 
5. Удалить запись    
6. Сохранить как     
0. Выйти из программы
\end{verbatim}
\end{tcolorbox}

Получили таблицу:

\begin{tcolorbox}
\begin{verbatim}
| ID   | байт Номер    | байт Марка    | байт Фамилия      | Осмотр     | 
| ---- | ------------- | ------------- | ----------------- | ---------- | 
| 0    | 8    m74y7d   | 5    Alpha    | 9    Podushkin    | Не пройден | 
| 1    | 8    d4rh75   | 4    Ford     | 7    Kamerov      | Не пройден | 
| 2    | 8    fruf45   | 3    BMW      | 8    Rozetkov     | Пройден    | 
| 3    | 8    f54h5    | 5    Tesla    | 7    Printov      | Не пройден | 
Нажмите любую клавишу для продолжения...
\end{verbatim}
\end{tcolorbox}
%===>>>

\newpage

%<<<===
\subsection{Индексный файл при сортировке}

В меню выбираю пункт сортировки.

\begin{tcolorbox}
\begin{verbatim}
Меню:
1. Открыть файл
2. Ввод данных       
3. Вывод данных      
4. Сортирова по полю 
5. Удалить запись    
6. Сохранить как     
0. Выйти из программы
\end{verbatim}
\end{tcolorbox}

Выбираю сортировку по номеру

\begin{tcolorbox}
\begin{verbatim}
1. Сортировка по номеру
2. Сортировка по марке
3. Сортировка по фамилии
4. Сортировка по осмотру
0. Выйти

По какой строке сортировать: 
\end{verbatim}
\end{tcolorbox}

Индексный файл теперь содержит информацию:

\begin{tcolorbox}
\begin{verbatim}
4
1
3
2
0
\end{verbatim}
\end{tcolorbox}

Выбираю в меню пункт вывода данных.

\begin{tcolorbox}
\begin{verbatim}
Меню:
1. Открыть файл
2. Ввод данных       
3. Вывод данных      
4. Сортирова по полю 
5. Удалить запись    
6. Сохранить как     
0. Выйти из программы
\end{verbatim}
\end{tcolorbox}

Получаем таблицу:

\begin{tcolorbox}
\begin{verbatim}
| ID   | байт Номер    | байт Марка    | байт Фамилия      | Осмотр     | 
| ---- | ------------- | ------------- | ----------------- | ---------- | 
| 0    | 8    85hf4ifj | 4    Lada     | 6    Banker       | Пройден    | 
| 1    | 8    d4rh75   | 4    Ford     | 7    Kamerov      | Не пройден | 
| 2    | 8    f54h5    | 5    Tesla    | 7    Printov      | Не пройден | 
| 3    | 8    fruf45   | 3    BMW      | 8    Rozetkov     | Пройден    | 
| 4    | 8    m74y7d   | 5    Alpha    | 9    Podushkin    | Не пройден | 
Нажмите любую клавишу для продолжения...
\end{verbatim}
\end{tcolorbox}

Для выхода в меню жму любую клавишу. В меню выбираю пункт сортировки по полю.

\begin{tcolorbox}
\begin{verbatim}
Меню:
1. Открыть файл
2. Ввод данных       
3. Вывод данных      
4. Сортирова по полю 
5. Удалить запись    
6. Сохранить как     
0. Выйти из программы
\end{verbatim}
\end{tcolorbox}

Выбираю сортировку по марке.

\begin{tcolorbox}
\begin{verbatim}
1. Сортировка по номеру
2. Сортировка по марке 
3. Сортировка по фамилии
4. Сортировка по осмотру     
0. Выйти

По какой строке сортировать: 
\end{verbatim}
\end{tcolorbox}

Индексный файл теперь содержит:

\begin{tcolorbox}
\begin{verbatim}
0
2
1
4
3
\end{verbatim}
\end{tcolorbox}

В меню выбираю пункт вывода данны.

\begin{tcolorbox}
\begin{verbatim}
Меню:
1. Открыть файл
2. Ввод данных       
3. Вывод данных      
4. Сортирова по полю 
5. Удалить запись    
6. Сохранить как     
0. Выйти из программы
\end{verbatim}
\end{tcolorbox}

Получили таблицу:

\begin{tcolorbox}
\begin{verbatim}
| ID   | байт Номер    | байт Марка    | байт Фамилия      | Осмотр     | 
| ---- | ------------- | ------------- | ----------------- | ---------- | 
| 0    | 8    m74y7d   | 5    Alpha    | 9    Podushkin    | Не пройден | 
| 1    | 8    fruf45   | 3    BMW      | 8    Rozetkov     | Пройден    | 
| 2    | 8    d4rh75   | 4    Ford     | 7    Kamerov      | Не пройден | 
| 3    | 8    85hf4ifj | 4    Lada     | 6    Banker       | Пройден    | 
| 4    | 8    f54h5    | 5    Tesla    | 7    Printov      | Не пройден | 
Нажмите любую клавишу для продолжения...
\end{verbatim}
\end{tcolorbox}

Для выхода в меню жму любую клавишу. В меню выбираю пункт сортировки поля.

\begin{tcolorbox}
\begin{verbatim}
Меню:
1. Открыть файл
2. Ввод данных       
3. Вывод данных      
4. Сортирова по полю 
5. Удалить запись    
6. Сохранить как     
0. Выйти из программы
\end{verbatim}
\end{tcolorbox}

Выбираем сортировку по фамилии

\begin{tcolorbox}
\begin{verbatim}
1. Сортировка по номеру
2. Сортировка по марке
3. Сортировка по фамилии
4. Сортировка по осмотру
0. Выйти

По какой строке сортировать: 
\end{verbatim}
\end{tcolorbox}

Индексный файл содержит:

\begin{tcolorbox}
\begin{verbatim}
4
1
0
3
2
\end{verbatim}
\end{tcolorbox}

В меню выбираю пункт вывода данных.

\begin{tcolorbox}
\begin{verbatim}
Меню:
1. Открыть файл
2. Ввод данных       
3. Вывод данных      
4. Сортирова по полю 
5. Удалить запись    
6. Сохранить как     
0. Выйти из программы
\end{verbatim}
\end{tcolorbox}

Получили таблицу:

\begin{tcolorbox}
\begin{verbatim}
| ID   | байт Номер    | байт Марка    | байт Фамилия      | Осмотр     | 
| ---- | ------------- | ------------- | ----------------- | ---------- | 
| 0    | 8    85hf4ifj | 4    Lada     | 6    Banker       | Пройден    | 
| 1    | 8    d4rh75   | 4    Ford     | 7    Kamerov      | Не пройден | 
| 2    | 8    m74y7d   | 5    Alpha    | 9    Podushkin    | Не пройден | 
| 3    | 8    f54h5    | 5    Tesla    | 7    Printov      | Не пройден | 
| 4    | 8    fruf45   | 3    BMW      | 8    Rozetkov     | Пройден    | 
Нажмите любую клавишу для продолжения...
\end{verbatim}
\end{tcolorbox}

Для выхода в меню жму любую клавишу. В меню выбираю пункт сортировка по полю.

\begin{tcolorbox}
\begin{verbatim}
Меню:
1. Открыть файл
2. Ввод данных       
3. Вывод данных      
4. Сортирова по полю 
5. Удалить запись    
6. Сохранить как     
0. Выйти из программы
\end{verbatim}
\end{tcolorbox}

Выбираю сортировку по осмотру.

\begin{tcolorbox}
\begin{verbatim}
1. Сортировка по номеру
2. Сортировка по марке       
3. Сортировка по фамилии     
4. Сортировка по осмотру     
0. Выйти

По какой строке сортировать: 
\end{verbatim}
\end{tcolorbox}

Индексный файл теперь содержит:

\begin{tcolorbox}
\begin{verbatim}
1
0
3
2
4
\end{verbatim}
\end{tcolorbox}

В меню выбираю пунт вывода данных

\begin{tcolorbox}
\begin{verbatim}
| ID   | байт Номер    | байт Марка    | байт Фамилия      | Осмотр     | 
| ---- | ------------- | ------------- | ----------------- | ---------- | 
| 0    | 8    d4rh75   | 4    Ford     | 7    Kamerov      | Не пройден | 
| 1    | 8    m74y7d   | 5    Alpha    | 9    Podushkin    | Не пройден | 
| 2    | 8    f54h5    | 5    Tesla    | 7    Printov      | Не пройден | 
| 3    | 8    fruf45   | 3    BMW      | 8    Rozetkov     | Пройден    | 
| 4    | 8    85hf4ifj | 4    Lada     | 6    Banker       | Пройден    | 
Нажмите любую клавишу для продолжения...
\end{verbatim}
\end{tcolorbox}
%===>>>

\newpage

%<<<===
\subsection{Индексный файл при создании поля}

В меню выбираю пункт ввода данных.

\begin{tcolorbox}
\begin{verbatim}
Меню:
1. Открыть файл
2. Ввод данных       
3. Вывод данных      
4. Сортирова по полю 
5. Удалить запись    
6. Сохранить как     
0. Выйти из программы
\end{verbatim}
\end{tcolorbox}

Ввожу размер номера

\begin{tcolorbox}
\begin{verbatim}
Размер строки номера:
\end{verbatim}
\end{tcolorbox}

Ввожу номер

\begin{tcolorbox}
\begin{verbatim}
Размер строки номера: 8  
Номер машины:
\end{verbatim}
\end{tcolorbox}

Ввожу размер строки марки.

\begin{tcolorbox}
\begin{verbatim}
Размер строки номера: 8  
Номер машины: fnuriigf

Размер строки марки машины:
\end{verbatim}
\end{tcolorbox}

Ввожу марку.

\begin{tcolorbox}
\begin{verbatim}
Размер строки номера: 8  
Номер машины: fnuriigf

Размер строки марки машины: 7
Марка машины:
\end{verbatim}
\end{tcolorbox}

Ввожу размер фамилии владельца

\begin{tcolorbox}
\begin{verbatim}
Размер строки номера: 8  
Номер машины: fnuriigf

Размер строки марки машины: 7
Марка машины: Turismo

Размер строки фамилии владельца:
\end{verbatim}
\end{tcolorbox}

Ввожу фамилию.

\begin{tcolorbox}
\begin{verbatim}
Размер строки номера: 8  
Номер машины: fnuriigf

Размер строки марки машины: 7
Марка машины: Turismo

Размер строки фамилии владельца: 6
Фамилия владельца:
\end{verbatim}
\end{tcolorbox}

Техосмотр прошел или нет?

\begin{tcolorbox}
\begin{verbatim}
Размер строки номера: 8  
Номер машины: fnuriigf

Размер строки марки машины: 7
Марка машины: Turismo

Размер строки фамилии владельца: 6
Фамилия владельца: Pultec

Осмотр:       
1. Есть осмотр
2. Нет осмотра
\end{verbatim}
\end{tcolorbox}

Попадаю в меню. Там выбираю пункт вывода данных.

\begin{tcolorbox}
\begin{verbatim}
Меню:
1. Открыть файл
2. Ввод данных
3. Вывод данных
4. Сортирова по полю
5. Удалить запись
6. Сохранить как
0. Выйти из программы
\end{verbatim}
\end{tcolorbox}
    
Получаю таблицу с добавленным полем.

\begin{tcolorbox}
\begin{verbatim}
| ID   | байт Номер    | байт Марка    | байт Фамилия      | Осмотр     | 
| ---- | ------------- | ------------- | ----------------- | ---------- | 
| 0    | 8    d4rh75   | 4    Ford     | 7    Kamerov      | Не пройден | 
| 1    | 8    m74y7d   | 5    Alpha    | 9    Podushkin    | Не пройден | 
| 2    | 8    f54h5    | 5    Tesla    | 7    Printov      | Не пройден | 
| 3    | 8    fruf45   | 3    BMW      | 8    Rozetkov     | Пройден    | 
| 4    | 8    85hf4ifj | 4    Lada     | 6    Banker       | Пройден    | 
| 5    | 8    fnuriigf | 7    Turismo  | 6    Pultec       | Не пройден | 
Нажмите любую клавишу для продолжения...
\end{verbatim}
\end{tcolorbox}

Смотрим индексный файл, а там добавился индексный

\begin{tcolorbox}
\begin{verbatim}
1
0
3
2
4
5
\end{verbatim}
\end{tcolorbox}

%===>>>
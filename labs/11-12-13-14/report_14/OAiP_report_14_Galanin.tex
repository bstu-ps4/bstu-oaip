\documentclass[12pt, a4paper]{article}

%<<<===Переменные для титульного листа
\newcommand{\kafedra}{ИИТ} % Кафедра X

\newcommand{\numberOfLab}{14} % Лабораторная работа №X
\newcommand{\semestr}{2} % За X семестр
\newcommand{\distiplina}{ОАиП} % По дисциплине <<X>>
\newcommand{\labTitle}{Модульное программирование} % Тема: <<X>>

% Выполнил
\newcommand{\kurs}{1} % Студент X курса
\newcommand{\group}{ПО-4 (1)} % группы X
\newcommand{\labAuthor}{Галанин П. И.} % X

% Проверил
\newcommand{\teacherStatus}{ст. преподаватель} % X
\newcommand{\teacher}{Хацкевич М. В.} % X

\newcommand{\labDate}{Брест, 2020} % X
%===>>>

%<<<===Стили
\usepackage{../../../sty/encoding} % кодировка
\usepackage{../../../sty/titlePage} % титульный лист
\usepackage{../../../sty/fields} % поля
\usepackage{../../../sty/imgs} % картинки
\usepackage{../../../sty/code} % исходный код
\usepackage{../../../sty/labData} % для лабораторной
%===>>>

\begin{document}

%<<<===Титульный лист
\maketitle
\setcounter{page}{2}
%===>>>

%<<<===Содержание
\renewcommand{\contentsname}{Содержание}
\tableofcontents
\newpage
%===>>>

%<<<===Тема ЛР
\labheading
%===>>>

%<<<===Цель
\labgoal{Изучить принципы модульного программирования в Си, ознакомиться с основными возможностями межмодульного взаимодействия.}
%===>>>

%<<<===Ход Работы
\labreport
%===>>>

%<<<===Условие
\section{Условие}
%В программу разработанную в лабораторной работе 12 внести следующие изменения и дополнения:

\begin{enumerate}
    \item Программа должна быть разделена на несколько модулей (например: работа с файлами, работа с интерфейом, обработка запросов к базе данных и т.п.).
    \item Взаимодействие модулей организовать при помощи вызова функций и переменных
    внешнего типа (extern).
\end{enumerate}

%===>>>

%<<<===Исполняемая программа
\section{Информация}
\subsection{Что входит в модуль}

\begin{itemize}
    \item *.c - файл с исходным кодом на Си. Содержит функцию.
    
\begin{lstlisting}[language=C, numbers=none, caption=primer.c]
#include "HelloWorld.h"

void HelloWorld()
{
    printf("Hello, World!\n");
}
\end{lstlisting}

    \item *.h - заголовочный файл. Содержит:
    \begin{itemize}

        \item запрет вопторного потключения

\begin{lstlisting}[language=C, numbers=none, caption=primer.c]
#ifndef _INT_HEAP_SORT_H_
    #define _INT_HEAP_SORT_H_

    //Содержимое заголовочного файла    
#endif
\end{lstlisting}

        \item объявление стандартных библиотек 
    
\begin{lstlisting}[language=C, numbers=none, caption=primer.h]
#include <stdlib.h>
\end{lstlisting}

        \item объявления локальный библиотек

\begin{lstlisting}[language=C, numbers=none, caption=*.h]
#include "int_swap/int_swap.h"
\end{lstlisting}

        \item прототипы функций

\begin{lstlisting}[language=C, numbers=none, caption=primer.h]
struct Node* create_int_node(struct Node*, int);
\end{lstlisting}

        \item структуры

\begin{lstlisting}[language=C, numbers=none, caption=primer.h]
struct Node*
{
    struct Node* next;
    int data;
};
\end{lstlisting}

    \end{itemize}
    \item *.tex - файл для верстки на Latex
    \item *.dwawio - файл, который содержит блоксхему функции
    \item *.png - блоксхема в виде картинки
\end{itemize}

\subsection{Принятая структура проекта}
{

\scriptsize

\begin{verbatim}
.
├── Makefile
├── data.bin
├── data.tsv
├── indices.txt
└── src
    ├── lab
    │   ├── lab.c
    │   ├── lab.drawio
    │   ├── lab.h
    │   ├── lab.png
    │   ├── lab.tex
    │   └── menu
    │       ├── del_data
    │       │   ├── del_data.c
    │       │   ├── del_data.drawio
    │       │   ├── del_data.h
    │       │   ├── del_data.png
    │       │   └── del_data.tex
    │       ├── input_data
    │       │   ├── input_data.c
    │       │   ├── input_data.drawio
    │       │   ├── input_data.h
    │       │   ├── input_data.png
    │       │   ├── input_data.tex
    │       │   ├── input_mark
    │       │   │   ├── input_mark.c
    │       │   │   ├── input_mark.drawio
    │       │   │   ├── input_mark.h
    │       │   │   ├── input_mark.png
    │       │   │   └── input_mark.tex
    │       │   ├── input_number
    │       │   │   ├── input_number.c
    │       │   │   ├── input_number.drawio
    │       │   │   ├── input_number.h
    │       │   │   ├── input_number.png
    │       │   │   └── input_number.tex
    │       │   ├── input_osmotr
    │       │   │   ├── input_osmotr.c
    │       │   │   ├── input_osmotr.drawio
    │       │   │   ├── input_osmotr.h
    │       │   │   ├── input_osmotr.png
    │       │   │   └── input_osmotr.tex
    │       │   └── input_surname
    │       │       ├── input_surname.c
    │       │       ├── input_surname.drawio
    │       │       ├── input_surname.h
    │       │       ├── input_surname.png
    │       │       └── input_surname.tex
    │       ├── menu.c
    │       ├── menu.drawio
    │       ├── menu.h
    │       ├── menu.png
    │       ├── menu.tex
    │       ├── open_file
    │       │   ├── get_file_pointer
    │       │   │   ├── get_file_pointer.c
    │       │   │   ├── get_file_pointer.drawio
    │       │   │   ├── get_file_pointer.h
    │       │   │   ├── get_file_pointer.png
    │       │   │   └── get_file_pointer.tex
    │       │   ├── if_file_not_founded
    │       │   │   ├── if_file_not_founded.c
    │       │   │   ├── if_file_not_founded.drawio
    │       │   │   ├── if_file_not_founded.h
    │       │   │   ├── if_file_not_founded.png
    │       │   │   └── if_file_not_founded.tex
    │       │   ├── open_file.c
    │       │   ├── open_file.h
    │       │   └── open_file.tex
    │       ├── out_data
    │       │   ├── out_data.c
    │       │   ├── out_data.drawio
    │       │   ├── out_data.h
    │       │   ├── out_data.png
    │       │   ├── out_data.tex
    │       │   └── write_table
    │       │       ├── write_table.c
    │       │       ├── write_table.drawio
    │       │       ├── write_table.h
    │       │       ├── write_table.png
    │       │       └── write_table.tex
    │       ├── save_as
    │       │   ├── file_out_data
    │       │   │   ├── file_out_data.c
    │       │   │   ├── file_out_data.drawio
    │       │   │   ├── file_out_data.h
    │       │   │   ├── file_out_data.png
    │       │   │   └── file_out_data.tex
    │       │   ├── save_as.c
    │       │   ├── save_as.drawio
    │       │   ├── save_as.h
    │       │   ├── save_as.png
    │       │   └── save_as.tex
    │       └── sort_data
    │           ├── mark_field_sort
    │           │   ├── mark_field_sort.c
    │           │   ├── mark_field_sort.drawio
    │           │   ├── mark_field_sort.h
    │           │   ├── mark_field_sort.png
    │           │   └── mark_field_sort.tex
    │           ├── number_field_sort
    │           │   ├── number_field_sort.c
    │           │   ├── number_field_sort.drawio
    │           │   ├── number_field_sort.h
    │           │   ├── number_field_sort.png
    │           │   ├── number_field_sort.tex
    │           │   └── numder_field_sort.o
    │           ├── osmotr_field_sort
    │           │   ├── osmotr_field_sort.c
    │           │   ├── osmotr_field_sort.drawio
    │           │   ├── osmotr_field_sort.h
    │           │   ├── osmotr_field_sort.png
    │           │   └── osmotr_field_sort.tex
    │           ├── sort_data.c
    │           ├── sort_data.drawio
    │           ├── sort_data.h
    │           ├── sort_data.png
    │           ├── sort_data.tex
    │           └── surname_field_sort
    │               ├── surname_field_sort.c
    │               ├── surname_field_sort.drawio
    │               ├── surname_field_sort.h
    │               ├── surname_field_sort.png
    │               └── surname_field_sort.tex
    ├── main.c
    ├── main.drawio
    ├── main.h
    ├── main.png
    ├── main.tex
    └── my_libs
        ├── clearConsole
        │   ├── clearConsole.c
        │   ├── clearConsole.drawio
        │   ├── clearConsole.h
        │   ├── clearConsole.png
        │   └── clearConsole.tex
        ├── encoding
        │   ├── encoding.c
        │   ├── encoding.drawio
        │   ├── encoding.h
        │   ├── encoding.png
        │   └── encoding.tex
        ├── getch
        │   ├── getch.c
        │   ├── getch.h
        │   └── getch.tex
        ├── int_swap
        │   ├── int_swap.c
        │   ├── int_swap.drawio
        │   ├── int_swap.h
        │   ├── int_swap.png
        │   └── int_swap.tex
        └── pause_console
            ├── pause_console.c
            ├── pause_console.drawio
            ├── pause_console.h
            ├── pause_console.png
            └── pause_console.tex

27 directories, 131 files
\end{verbatim}

}

\subsection{Сборка модулей}

Для сборки используем Makefile. В терминал вводим команду "make", которая исходные файлы *.с скомпилирует в объектные, после чего соберёт объектные файлы в исходный файл zzz\_pear\_test, который можем запустить командой "./zzz\_pear\_test".

\lstinputlisting[
    language=make,
    name=Makefile
]{../13/Makefile}
%===>>>

%<<<===Вывод ЛР
\labconclusion{Изучили принципы модульного программирования в Си.}
%===>>>

%<<<===Литература
\newpage
\begin{thebibliography}{3}
    \bibitem{}
    БрГТУ.ПОИТ. Лабораторная работа "Модульное программирование"
\end{thebibliography}

\textbf{Ссылки}
\begin{itemize}
    \item Репозиторий с исходным кодом\\
    https://github.com/bstu-ps4/BSTU-OAiP
\end{itemize}
%===>>>

\end{document}
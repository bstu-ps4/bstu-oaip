\documentclass[12pt, a4paper]{article}

\newcommand{\kafedra}{ИИТ} % Кафедра X

\newcommand{\numberOfLab}{11/12} % Лабораторная работа №X
\newcommand{\semestr}{2} % За X семестр
\newcommand{\distiplina}{ОАиП} % По дисциплине <<X>>
\newcommand{\labTitle}{Структуры, перечисления, объединения. / Бинарные и текстовые файлы} % Тема: <<X>>

% Выполнил
\newcommand{\kurs}{1} % Студент X курса
\newcommand{\group}{ПО-4 (1)} % группы X
\newcommand{\labAuthor}{Галанин П. И.} % X

% Проверил
\newcommand{\teacherStatus}{ст. преподаватель} % X
\newcommand{\teacher}{Хацкевич М. В.} % X

\newcommand{\labDate}{Брест, 2020} % X

\usepackage{../../../sty/encoding} % кодировка
\usepackage{../../../sty/titlePage} % титульный лист
\usepackage{../../../sty/fields} % поля
\usepackage{../../../sty/imgs} % картинки
\usepackage{../../../sty/code} % исходный код
\usepackage{../../../sty/labData} % для лабораторной

\begin{document}

% = = = = = Титульный лист
\maketitle
\setcounter{page}{2}

% = = = = = Содержание
\renewcommand{\contentsname}{Содержание}
\tableofcontents
\newpage

% = = = = = Тема ЛР
\labheading

% = = = = = Цель
\labgoal{
\begin{enumerate}
    \item Изучить синтаксис и правила работы со структурами. Реализовать программу с применением структур, перечислений и объединений.
    \item Изучить принципы программирования с использованием бинарных файлов. Ознакомиться с основными функциями в Си для работы с бинарными файлами.
\end{enumerate}
}

% = = = = = Ход Работы
\labreport

% = = = = = Структура проекта
\section{Структура проекта}
        \begin{verbatim}
.
├── Makefile
└── src
    ├── lab
    │   ├── lab.c     
    │   ├── lab.drawio
    │   ├── lab.h     
    │   ├── lab.png   
    │   ├── lab.tex   
    │   └── menu
    │       ├── del_data
    │       │   ├── del_data.c     
    │       │   ├── del_data.drawio
    │       │   ├── del_data.h     
    │       │   ├── del_data.png
    │       │   └── del_data.tex
    │       ├── input_data
    │       │   ├── input_data.c
    │       │   ├── input_data.drawio
    │       │   ├── input_data.h
    │       │   ├── input_data.png
    │       │   └── input_data.tex
    │       ├── menu.c
    │       ├── menu.drawio
    │       ├── menu.h
    │       ├── menu.png
    │       ├── menu.tex
    │       ├── out_data
    │       │   ├── out_data.c
    │       │   ├── out_data.h
    │       │   ├── out_data.png
    │       │   └── out_data.tex
    │       └── sort_data
    │           ├── sort_data-get_sorted_array.png
    │           ├── sort_data-sort_data.png
    │           ├── sort_data.c
    │           ├── sort_data.drawio
    │           ├── sort_data.h
    │           ├── sort_data.tex
    │           ├── sort_data_by_mark_field
    │           │   ├── sort_data_by_mark_field.c
    │           │   ├── sort_data_by_mark_field.drawio
    │           │   ├── sort_data_by_mark_field.h
    │           │   ├── sort_data_by_mark_field.png
    │           │   └── sort_data_by_mark_field.tex
    │           ├── sort_data_by_osmotr_field
    │           │   ├── sort_data_by_osmotr_field.c
    │           │   ├── sort_data_by_osmotr_field.drawio
    │           │   ├── sort_data_by_osmotr_field.h
    │           │   ├── sort_data_by_osmotr_field.png
    │           │   └── sort_data_by_osmotr_field.tex
    │           ├── sort_data_by_surname_field
    │           │   ├── sort_data_by_surname_field.c
    │           │   ├── sort_data_by_surname_field.drawio
    │           │   ├── sort_data_by_surname_field.h
    │           │   ├── sort_data_by_surname_field.png
    │           │   └── sort_data_by_surname_field.tex
    │           └── sort_data_in_number_field
    │               ├── sort_data_in_number_field.c
    │               ├── sort_data_in_number_field.drawio
    │               ├── sort_data_in_number_field.h
    │               ├── sort_data_in_number_field.png
    │               └── sort_data_in_number_field.tex
    ├── main.c
    ├── main.drawio
    ├── main.h
    ├── main.png
    ├── main.tex
    └── my_libs
        ├── clearConsole
        │   ├── clearConsole.c
        │   ├── clearConsole.drawio
        │   ├── clearConsole.h
        │   ├── clearConsole.png
        │   └── clearConsole.tex
        ├── encoding
        │   ├── encoding.c
        │   ├── encoding.drawio
        │   ├── encoding.h
        │   ├── encoding.png
        │   └── encoding.tex
        ├── getch
        │   ├── getch.c
        │   ├── getch.h
        │   └── getch.tex
        └── pause_console
            ├── pause_console.c
            ├── pause_console.drawio
            ├── pause_console.h
            ├── pause_console.png
            └── pause_console.tex

16 directories, 74 files
\end{verbatim}

% = = = = = Основа
\newpage
\section{Основа}
    % = = = = = Условие
    \subsection{Условие}
        В программу разработанную в лабораторной работе 12 внести следующие изменения и дополнения:

\begin{enumerate}
    \item При запуске программы данные читаются в массив структур из файла, при добавлении новой записи в массив структур в файл должна дописываться новая запись, без изменения остальных записей.
    \item Все изменения (изменения полей записи, удаление записи) – сохраняются в файле при помощи перезаписи содержимого всего файла.
    \item Сортировка должна выполняться по двум полям на выбор при помощи создания индексных файлов. Содержимое индексного файла переписывается в случае изменения значения ключевого поля (поля, по которому выполняется сортировка) или в случае удаления, добавления записей.
\end{enumerate}


    % = = = = = Исходный код
    \newpage
    \subsection{Исходный код}
        \documentclass[12pt,a4paper]{article}

\usepackage{src/encoding} % кодировка
\usepackage{src/titlePage} % титульный лист
\usepackage{src/fields} % поля
\usepackage{src/imgs} % картинки

\begin{document}

% титульный лист
\maketitle
\setcounter{page}{2}

%
.
\newpage

% содержание
\renewcommand{\contentsname}{Содержание}
\tableofcontents
\newpage

% контент
\section{2020-03-12}

\input{src/2020-03-12/1.tex}
\newpage
\input{src/2020-03-12/2.tex}
\newpage
\input{src/2020-03-12/3.tex}
\newpage
\input{src/2020-03-12/4.tex}
\newpage
\input{src/2020-03-12/5.tex}
\newpage
\input{src/2020-03-12/6.tex}
\newpage
\input{src/2020-03-12/7.tex}
\newpage
\newpage
\section{2020-03-12}

\input{src/2020-03-12/1.tex}
\newpage
\input{src/2020-03-12/2.tex}
\newpage
\input{src/2020-03-12/3.tex}
\newpage
\input{src/2020-03-12/4.tex}
\newpage
\input{src/2020-03-12/5.tex}
\newpage
\input{src/2020-03-12/6.tex}
\newpage
\input{src/2020-03-12/7.tex}
\newpage
\newpage
\section{2020-03-12}

\input{src/2020-03-12/1.tex}
\newpage
\input{src/2020-03-12/2.tex}
\newpage
\input{src/2020-03-12/3.tex}
\newpage
\input{src/2020-03-12/4.tex}
\newpage
\input{src/2020-03-12/5.tex}
\newpage
\input{src/2020-03-12/6.tex}
\newpage
\input{src/2020-03-12/7.tex}
\newpage
\newpage

\end{document}

\subsection{clearConsole()}

Блок-схема на рисунке \ref{fig:clearConsole}.

\begin{figure}[h]
    \center{
        \includegraphics[]{../13/src/my_libs/clearConsole/clearConsole.png}
    }
    \caption{clearConsole()}
    \label{fig:clearConsole}
\end{figure}

\lstinputlisting[
    language=C,
    name=clearConsole.h
]{../13/src/my_libs/clearConsole/clearConsole.h}

\lstinputlisting[
    language=C,
    name=clearConsole.c
]{../13/src/my_libs/clearConsole/clearConsole.c}

\newpage
\usepackage[utf8]{inputenc}
\usepackage[russian]{babel}
\usepackage[T2A]{fontenc}
\usepackage{ucs}

\subsection{getch()}

\lstinputlisting[
    language=C,
    name=getch.h
]{../13/src/my_libs/getch/getch.h}

\lstinputlisting[
    language=C,
    name=getch.h
]{../13/src/my_libs/getch/getch.c}

\newpage
\subsubsection{pause\_console()}

Блок-схема на рисунке \ref{fig:pause_console}.

\begin{figure}[h]
    \center{
        \includegraphics[]{../13/src/my_libs/pause_console/pause_console.png}
    }
    \caption{pause\_console()}
    \label{fig:pause_console}
\end{figure}

\lstinputlisting[
    language=C,
    name=pause\_console.h
]{../13/src/my_libs/pause_console/pause_console.h}

\lstinputlisting[
    language=C,
    name=pause\_console.c
]{../13/src/my_libs/pause_console/pause_console.c}

\newpage

\subsubsection{lab()}

Блок-схема на рисунке \ref{fig:lab}.

\begin{figure}[h]
    \center{
        \includegraphics[]{../13/src/lab/lab.png}
    }
    \caption{lab()}
    \label{fig:lab}
\end{figure}

\lstinputlisting[
    language=C,
    name=lab.h
]{../13/src/lab/lab.h}

\lstinputlisting[
    language=C,
    name=lab.c
]{../13/src/lab/lab.c}

\newpage
\subsection{menu()}

Блок-схема на рисунке \ref{fig:menu}.

\begin{figure}[h]
    \center{
        \includegraphics[width=16cm]{../13/src/lab/menu/menu.png}
    }
    \caption{menu()}
    \label{fig:menu}
\end{figure}

\lstinputlisting[
    language=C,
    name=menu.h
]{../13/src/lab/menu/menu.h}

\lstinputlisting[
    language=C,
    name=menu.c
]{../13/src/lab/menu/menu.c}

\newpage
\subsubsection{del\_data()}

Блок-схема на рисунке \ref{fig:del_data}.

\begin{figure}[p]
    \center{
        \includegraphics[]{../13/src/lab/menu/del_data/del_data.png}
    }
    \caption{del\_data()}
    \label{fig:del_data}
\end{figure}

\lstinputlisting[
    language=C,
    name=del\_data.h
]{../13/src/lab/menu/del_data/del_data.h}

\lstinputlisting[
    language=C,
    name=del\_data.c
]{../13/src/lab/menu/del_data/del_data.c}

\newpage

\subsection{input\_data()}

Блок-схема на рисунке \ref{fig:input_data}.

\begin{figure}[p]
    \center{
        \includegraphics[width=16cm]{../13/src/lab/menu/input_data/input_data.png}
    }
    \caption{input\_data()}
    \label{fig:input_data}
\end{figure}

\lstinputlisting[
    language=C,
    name=input\_data.h
]{../13/src/lab/menu/input_data/input_data.h}

\lstinputlisting[
    language=C,
    name=input\_data.c
]{../13/src/lab/menu/input_data/input_data.c}

\newpage
\subsubsection{input\_number()}

Блок-схема на рисунке \ref{fig:input_number}.

\begin{figure}[p]
    \center{
        \includegraphics[width=16cm]{../13/src/lab/menu/input_data/input_number/input_number.png}
    }
    \caption{input\_number()}
    \label{fig:input_number}
\end{figure}

\lstinputlisting[
    language=C,
    name=input\_number.h
]{../13/src/lab/menu/input_data/input_number/input_number.h}

\lstinputlisting[
    language=C,
    name=input\_number.c
]{../13/src/lab/menu/input_data/input_number/input_number.c}

\newpage
\subsection{input\_mark()}

Блок-схема на рисунке \ref{fig:input_mark}.

\begin{figure}[p]
    \center{
        \includegraphics[width=16cm]{../13/src/lab/menu/input_data/input_mark/input_mark.png}
    }
    \caption{input\_mark()}
    \label{fig:input_mark}
\end{figure}

\lstinputlisting[
    language=C,
    name=input\_mark.h
]{../13/src/lab/menu/input_data/input_mark/input_mark.h}

\lstinputlisting[
    language=C,
    name=input\_mark.c
]{../13/src/lab/menu/input_data/input_mark/input_mark.c}

\newpage
\subsection{input\_surname()}

Блок-схема на рисунке \ref{fig:input_surname}.

\begin{figure}[p]
    \center{
        \includegraphics[width=16cm]{../13/src/lab/menu/input_data/input_surname/input_surname.png}
    }
    \caption{input\_surname()}
    \label{fig:input_surname}
\end{figure}

\lstinputlisting[
    language=C,
    name=input\_surname.h
]{../13/src/lab/menu/input_data/input_surname/input_surname.h}

\lstinputlisting[
    language=C,
    name=input\_surname.c
]{../13/src/lab/menu/input_data/input_surname/input_surname.c}

\newpage
\subsubsection{input\_osmotr()}

Блок-схема на рисунке \ref{fig:input_osmotr}.

\begin{figure}[p]
    \center{
        \includegraphics[width=16cm]{../13/src/lab/menu/input_data/input_osmotr/input_osmotr.png}
    }
    \caption{input\_osmotr()}
    \label{fig:input_osmotr}
\end{figure}

\lstinputlisting[
    language=C,
    name=input\_osmotr.h
]{../13/src/lab/menu/input_data/input_osmotr/input_osmotr.h}

\lstinputlisting[
    language=C,
    name=input\_osmotr.c
]{../13/src/lab/menu/input_data/input_osmotr/input_osmotr.c}

\newpage

\subsection{out\_data()}

Блок-схема на рисунке \ref{fig:out_data}.

\begin{figure}[p]
    \center{
        \includegraphics[]{../13/src/lab/menu/out_data/out_data.png}
    }
    \caption{out\_data()}
    \label{fig:out_data}
\end{figure}

\lstinputlisting[
    language=C,
    name=out\_data.h
]{../13/src/lab/menu/out_data/out_data.h}

\lstinputlisting[
    language=C,
    name=out\_data.c
]{../13/src/lab/menu/out_data/out_data.c}

\newpage

\subsubsection{sort\_data()}

Блок-схема на рисунке \ref{fig:sort_data}.

\begin{figure}[p]
    \center{
        \includegraphics[]{../13/src/lab/menu/sort_data/sort_data.png}
    }
    \caption{sort\_data()}
    \label{fig:sort_data}
\end{figure}

\lstinputlisting[
    language=C,
    name=sort\_data.h
]{../13/src/lab/menu/sort_data/sort_data.h}

\lstinputlisting[
    language=C,
    name=sort\_data.c
]{../13/src/lab/menu/sort_data/sort_data.c}

\newpage
\subsection{get\_sorted\_array()}

Блок-схема на рисунке \ref{fig:get_sorted_array}.

\begin{figure}[p]
    \center{
        \includegraphics[]{../13/src/lab/menu/sort_data/get_sorted_array/get_sorted_array.png}
    }
    \caption{get\_sorted\_array()}
    \label{fig:get_sorted_array}
\end{figure}

\lstinputlisting[
    language=C,
    name=get\_sorted\_array.h
]{../13/src/lab/menu/sort_data/get_sorted_array/get_sorted_array.h}

\lstinputlisting[
    language=C,
    name=get\_sorted\_array.c
]{../13/src/lab/menu/sort_data/get_sorted_array/get_sorted_array.c}

\newpage
\subsection{sort\_data\_by\_mark\_field()}

\lstinputlisting[
	language=C,
	name=sort\_data\_by\_mark\_field.h
]{../13/src/lab/menu/sort_data/sort_data_by_mark_field/sort_data_by_mark_field.h}

\lstinputlisting[
	language=C,
	name=sort\_data\_by\_mark\_field.c
]{../13/src/lab/menu/sort_data/sort_data_by_mark_field/sort_data_by_mark_field.c}

\newpage
\subsection{sort\_data\_by\_osmotr\_field()}

Блок-схема на рисунке \ref{fig:sort_data_by_osmotr_field}.

\begin{figure}[p]
    \center{
        \includegraphics[]{../13/src/lab/menu/sort_data/sort_data_by_osmotr_field/sort_data_by_osmotr_field.png}
    }
    \caption{sort\_data\_by\_osmotr\_field()}
    \label{fig:sort_data_by_osmotr_field}
\end{figure}

\lstinputlisting[
	language=C,
	name=sort\_data\_by\_osmotr\_field.h
]{../13/src/lab/menu/sort_data/sort_data_by_osmotr_field/sort_data_by_osmotr_field.h}

\lstinputlisting[
	language=C,
	name=sort\_data\_by\_osmotr\_field.c
]{../13/src/lab/menu/sort_data/sort_data_by_osmotr_field/sort_data_by_osmotr_field.c}

\newpage
\subsection{sort\_data\_by\_surname\_field()}

\lstinputlisting[
	language=C,
	name=sort\_data\_by\_surname\_field.h
]{../13/src/lab/menu/sort_data/sort_data_by_surname_field/sort_data_by_surname_field.h}

\lstinputlisting[
	language=C,
	name=sort\_data\_by\_surname\_field.c
]{../13/src/lab/menu/sort_data/sort_data_by_surname_field/sort_data_by_surname_field.c}

\newpage
\subsection{sort\_data\_in\_number\_field()}

\lstinputlisting[
	language=C,
	name=sort\_data\_in\_number\_field.h
]{../13/src/lab/menu/sort_data/sort_data_in_number_field/sort_data_in_number_field.h}

\lstinputlisting[
	language=C,
	name=sort\_data\_in\_number\_field.c
]{../13/src/lab/menu/sort_data/sort_data_in_number_field/sort_data_in_number_field.c}

\newpage

    % = = = = = Исполняемая программа
    \newpage
    \subsection{Исполняемая программа}
        Запускаю программу

\begin{tcolorbox}
\begin{verbatim}
Меню:
1. Открыть файл     
2. Ввод данных      
3. Вывод данных     
4. Сортирова по полю
5. Удалить запись    
6. Сохранить как     
0. Выйти из программы
\end{verbatim}
\end{tcolorbox}

Выбираю пункт 2

\begin{tcolorbox}
\begin{verbatim}
Размер строки номера: 
\end{verbatim}
\end{tcolorbox}

Ввожу размер строки номера

\begin{tcolorbox}
\begin{verbatim}
Размер строки номера: 8  
Номер машины: 
\end{verbatim}
\end{tcolorbox}

Ввожу сам номер

\begin{tcolorbox}
\begin{verbatim}
Размер строки номера: 8  
Номер машины: fruf45

Размер строки марки машины: 
\end{verbatim}
\end{tcolorbox}

Ввожу размер строки марки 

\begin{tcolorbox}
\begin{verbatim}
Размер строки номера: 8
Номер машины: fruf45

Размер строки марки машины: 3
Марка машины: 
\end{verbatim}
\end{tcolorbox}

Ввожу марку машины

\begin{tcolorbox}
\begin{verbatim}
Размер строки номера: 8
Номер машины: fruf45

Размер строки марки машины: 3
Марка машины: BMW

Размер строки фамилии владельца: 
\end{verbatim}
\end{tcolorbox}

Ввожу размер фамилии

\begin{tcolorbox}
\begin{verbatim}
Размер строки номера: 8
Номер машины: fruf45

Размер строки марки машины: 3
Марка машины: BMW

Размер строки фамилии владельца: 10
Фамилия владельца:
\end{verbatim}
\end{tcolorbox}

Ввожу фамилию владельца

\begin{tcolorbox}
\begin{verbatim}
Размер строки номера: 8
Номер машины: fruf45

Размер строки марки машины: 6
Марка машины: БМВ

Размер строки фамилии владельца: 10
Фамилия владельца: Rozetkov

Осмотр:       
1. Есть осмотр
2. Нет осмотра
\end{verbatim}
\end{tcolorbox}

Ввожу 1. (Если нажать другую клавишу - повторит цикл ввода осмотра)

Открылось главное меню. Выбираю 3-ий пункт.

\begin{tcolorbox}
\begin{verbatim}
| ID   | байт / Номер    | байт / Марка    | байт / Фамилия      | Осмотр     |
| ---- | ---- / -------- | ---- / -------- | ---- / ------------ | ---------- |
| 0    | 8    / fruf45   | 3    / BMW      | 10   / Rozetkov     | Пройден    |
Нажмите любую клавишу для продолжения...
\end{verbatim}
\end{tcolorbox}

Жму любую клавишу. Попадаю в главное меню. Выбираю пункт 1. Заполняю данные... Выбираю пункт 2 для просмотра.

\begin{tcolorbox}
\begin{verbatim}
| ID   | байт / Номер    | байт / Марка    | байт / Фамилия      | Осмотр     |
| ---- | ---- / -------- | ---- / -------- | ---- / ------------ | ---------- |
| 0    | 8    / fruf45   | 3    / BMW      | 10   / Rozetkov     | Пройден    |
| 1    | 8    / m74y7d   | 10   / Alpha    | 10   / Podushkin    | Не пройден |
| 2    | 8    / d4rh75   | 4    / Ford     | 10   / Kamerov      | Не пройден |
| 3    | 8    / diejd    | 4    / Lada     | 10   / Zaraydkov    | Не пройден |
Нажмите любую клавишу для продолжения...
\end{verbatim}
\end{tcolorbox}

Нажимаю любую клавишу. Попадаю в главное меню. Выбираю пункт 4.

\begin{tcolorbox}
\begin{verbatim}
1. Сортировка по номеру
2. Сортировка по марке  
3. Сортировка по фамилии
4. Сортировка по осмотру
0. Выйти

По какой строке сортировать: 
\end{verbatim}
\end{tcolorbox}

Выбираю пункт 1. Попадаю в главное меню. Нажимаю пункт 3 для вывода таблицы.

\begin{tcolorbox}
\begin{verbatim}
| ID   | байт / Номер    | байт / Марка    | байт / Фамилия      | Осмотр     |
| ---- | ---- / -------- | ---- / -------- | ---- / ------------ | ---------- |
| 0    | 8    / d4rh75   | 4    / Ford     | 10   / Kamerov      | Не пройден |
| 1    | 8    / diejd    | 4    / Lada     | 10   / Zaraydkov    | Не пройден |
| 2    | 8    / fruf45   | 3    / BMW      | 10   / Rozetkov     | Пройден    |
| 3    | 8    / m74y7d   | 10   / Alpha    | 10   / Podushkin    | Не пройден |
Нажмите любую клавишу для продолжения...
\end{verbatim}
\end{tcolorbox}

Жму любую клавишу. Попадаю в главное меню. Выбираю пункт 4 для сортировки. Выбираю пункт 2. Попадаю в главное меню. Выбираю пункт 3 для вывода таблицы.

\begin{tcolorbox}
\begin{verbatim}
| ID   | байт / Номер    | байт / Марка    | байт / Фамилия      | Осмотр     |
| ---- | ---- / -------- | ---- / -------- | ---- / ------------ | ---------- |
| 0    | 8    / m74y7d   | 10   / Alpha    | 10   / Podushkin    | Не пройден |
| 1    | 8    / fruf45   | 3    / BMW      | 10   / Rozetkov     | Пройден    |
| 2    | 8    / d4rh75   | 4    / Ford     | 10   / Kamerov      | Не пройден |
| 3    | 8    / diejd    | 4    / Lada     | 10   / Zaraydkov    | Не пройден |
Нажмите любую клавишу для продолжения...
\end{verbatim}
\end{tcolorbox}

Жму любую клавишу. Попадаю в главное меню. Выбираю пункт 4 для сортировки. Выбираю пункт 3. Попадаю в главное меню. Выбираю пункт 3 для вывода таблицы.

\begin{tcolorbox}
\begin{verbatim}
| ID   | байт / Номер    | байт / Марка    | байт / Фамилия      | Осмотр     |
| ---- | ---- / -------- | ---- / -------- | ---- / ------------ | ---------- |
| 0    | 8    / d4rh75   | 4    / Ford     | 10   / Kamerov      | Не пройден |
| 1    | 8    / m74y7d   | 10   / Alpha    | 10   / Podushkin    | Не пройден |
| 2    | 8    / fruf45   | 3    / BMW      | 10   / Rozetkov     | Пройден    |
| 3    | 8    / diejd    | 4    / Lada     | 10   / Zaraydkov    | Не пройден |
Нажмите любую клавишу для продолжения...
\end{verbatim}
\end{tcolorbox}

Жму любую клавишу. Попадаю в главное меню. Выбираю пункт 4 для сортировки. Выбираю пункт 4. Попадаю в главное меню. Выбираю пункт 3 для вывода таблицы.

\begin{tcolorbox}
\begin{verbatim}
| ID   | байт / Номер    | байт / Марка    | байт / Фамилия      | Осмотр     |
| ---- | ---- / -------- | ---- / -------- | ---- / ------------ | ---------- |
| 0    | 8    / m74y7d   | 10   / Alpha    | 10   / Podushkin    | Не пройден |
| 1    | 8    / d4rh75   | 4    / Ford     | 10   / Kamerov      | Не пройден |
| 2    | 8    / diejd    | 4    / Lada     | 10   / Zaraydkov    | Не пройден |
| 3    | 8    / fruf45   | 3    / BMW      | 10   / Rozetkov     | Пройден    |
Нажмите любую клавишу для продолжения...
\end{verbatim}
\end{tcolorbox}

Жму любую клавишу. Попадаю в главное меню. Выбираю пункт 5 для удаления элемента.

\begin{tcolorbox}
\begin{verbatim}
Какой элемент удалить:
\end{verbatim}
\end{tcolorbox}

Жму 2. Жму Enter. Попадаю в главное меню. Выбираю пункт 3 для вывода таблицы.

\begin{tcolorbox}
\begin{verbatim}
| ID   | байт / Номер    | байт / Марка    | байт / Фамилия      | Осмотр     |
| ---- | ---- / -------- | ---- / -------- | ---- / ------------ | ---------- |
| 0    | 8    / m74y7d   | 10   / Alpha    | 10   / Podushkin    | Не пройден |
| 1    | 8    / d4rh75   | 4    / Ford     | 10   / Kamerov      | Не пройден |
| 2    | 8    / fruf45   | 3    / BMW      | 10   / Rozetkov     | Пройден    |
Нажмите любую клавишу для продолжения...
\end{verbatim}
\end{tcolorbox}

Нажимаю любую клавишу. Переходу в главное меню. Выбираю пункт 5. Ввожу 55. Попадаю в главное меню. Выбираю пункт 3 для вывода таблицы.

\begin{tcolorbox}
\begin{verbatim}
| ID   | байт / Номер    | байт / Марка    | байт / Фамилия      | Осмотр     |
| ---- | ---- / -------- | ---- / -------- | ---- / ------------ | ---------- |
| 0    | 8    / m74y7d   | 10   / Alpha    | 10   / Podushkin    | Не пройден |
| 1    | 8    / d4rh75   | 4    / Ford     | 10   / Kamerov      | Не пройден |
| 2    | 8    / fruf45   | 3    / BMW      | 10   / Rozetkov     | Пройден    |
Нажмите любую клавишу для продолжения...
\end{verbatim}
\end{tcolorbox}

Таблица не поменялась, так как нет такого ID.

Нажимаю 0, чтобы выйти из программы.

    % = = = = = Вывод ЛР
    \labconclusion{}

% = = = = =
\newpage
\section{Бинарные и текстовые файлы}
    % = = = = = Условие
    \subsection{Условие}
        В программу разработанную в лабораторной работе 10 добавить чтение и сохранение данных массива структур при помощи бинарных файлов следующим образом:
\begin{itemize}
    \item При первом запуске программы должен создаваться бинарный или текстовый файл на выбор пользователя для хранения данных из массива структур.
    \item При добавлении новой записи в массив структур в файл должна дописываться новая запись, без изменения остальных записей.
    \item При повторном запуске программы, если файл уже существует, то информация в массив структур должна читаться из этого файла. Если файл отсутствует, то он должен создаваться (см. Пункт 1).
    \item Все изменения (сортировка, изменения полей записи, удаление записи) – сохраняются в файле при помощи полной перезаписи содержимого.
    \item Сделать вывод о том, какие преимущества использования конкретного типа файлов (бинарные или текстовые) в решаемой вами задаче.
\end{itemize}


    % = = = = = Исходный код
    \newpage
    \subsection{Исходный код}
        \subsubsection{open\_file()}

%Блок-схема на рисунке \ref{fig:open\_file}.

%\begin{figure}[p]
%    \center{
%        \includegraphics[]{../13/src/lab/menu/open_file/open_file.png}
%    }
%    \caption{open\_file()}
%    \label{fig:open_file}
%\end{figure}

\lstinputlisting[
    language=C,
    name=open\_file.h
]{../13/src/lab/menu/open_file/open_file.h}

\lstinputlisting[
    language=C,
    name=open\_file.c
]{../13/src/lab/menu/open_file/open_file.c}

\newpage
\subsubsection{save\_as()}

Блок-схема на рисунке \ref{fig:save_as}.

\begin{figure}[p]
    \center{
        \includegraphics[]{../13/src/lab/menu/save_as/save_as.png}
    }
    \caption{save\_as()}
    \label{fig:save_as}
\end{figure}

\lstinputlisting[
    language=C,
    name=save_as.h
]{../13/src/lab/menu/save_as/save_as.h}

\lstinputlisting[
    language=C,
    name=save_as.c
]{../13/src/lab/menu/save_as/save_as.c}

\newpage
\subsubsection{file\_out\_data()}

%Блок-схема на рисунке \ref{fig:file_out_data}.

%\begin{figure}[p]
%    \center{
%        \includegraphics[]{../13/src/lab/menu/save_as/file_out_data/file_out_data.png}
%    }
%    \caption{file\_out\_data()}
%    \label{fig:file_out_data}
%\end{figure}

\lstinputlisting[
    language=C,
    name=file\_out\_data.h
]{../13/src/lab/menu/save_as/file_out_data/file_out_data.h}

\lstinputlisting[
    language=C,
    name=file\_out\_data.c
]{../13/src/lab/menu/save_as/file_out_data/file_out_data.c}

\newpage


    % = = = = = Исполняемая программа
    \newpage
    \subsection{Исполняемая программа}
        % = = = = = Сохранение файла как *.tsv

\subsubsection{Сохранение файла как *.tsv}

Из меню выбираю бункт 3, для вывода таблицы в консоль.

\begin{tcolorbox}
\begin{verbatim}
| ID   | байт Номер    | байт Марка    | байт Фамилия      | Осмотр     |
| ---- | ---- -------- | ---- -------- | ---- ------------ | ---------- |
| 0    | 8    m74y7d   | 10   Alpha    | 10   Podushkin    | Не пройден |
| 1    | 8    d4rh75   | 4    Ford     | 10   Kamerov      | Не пройден |
| 2    | 8    fruf45   | 3    BMW      | 10   Rozetkov     | Пройден    |
Нажмите любую клавишу для продолжения...
\end{verbatim}
\end{tcolorbox}

Нажимаю любую клавишу. Выбираю из меню пункт 6, чтобы сохранить файл.

\begin{tcolorbox}
\begin{verbatim}
Меню:
1. Сохранить как tsv файл
2. Сохранить как bin файл
0. Выйти
\end{verbatim}
\end{tcolorbox}

Выбираю пункт 1, чтобы сохранить как tsv файл. TSV - Tab Separated Values файл. Открыть таблицу можно, например, в Libre Office Calc (Скриншот на рисунке \ref{fig:data-tsv}). Также можно такие файлы просматривать на GitHub (Скриншот на рисунке \ref{fig:data-tsv-on-github}).

Если открыть файл, то там:
\begin{tcolorbox}
\begin{verbatim}
0	8	m74y7d	10	Alpha	10	Podushkin	1
1	8	d4rh75	4	Ford	10	Kamerov	1
2	8	fruf45	3	BMW	10	Rozetkov	0
\end{verbatim}
\end{tcolorbox}

\begin{figure}[h]
    \center{
        \includegraphics[width=16cm]{../pics/data-tsv-on-libre-office-calc.png}
    }
    \caption{data.tsv}
    \label{fig:data-tsv}
\end{figure}

\begin{figure}[h]
    \center{
        \includegraphics[width=16cm]{../pics/data-tsv-on-github.png}
    }
    \caption{data.tsv}
    \label{fig:data-tsv-on-github}
\end{figure}

% = = = = = Открытие файла

\subsubsection{Открытие файла}

Проредактировав поля в Libre Office Calc, файл содержит:

\begin{tcolorbox}
\begin{verbatim}
0	8	m74y7d	5	Alpha	9	Podushkin	1
1	8	d4rh75	4	Ford	7	Kamerov	1
2	8	fruf45	3	BMW	8	Rozetkov	0
3	8	f54h5	5	Tesla	7	Printov	1
4	8	85hf4ifj	4	Lada	6	Banker	0
\end{verbatim}
\end{tcolorbox}

Захожу в свою программу. Из меню выбираю пункт 1, чтобы открыть файл.

\begin{tcolorbox}
\begin{verbatim}
Размер пути файла: 
\end{verbatim}
\end{tcolorbox}

Ввожу размер пути файла.

\begin{tcolorbox}
\begin{verbatim}
Размер пути файла: 10
Какой файл открыть:     
\end{verbatim}
\end{tcolorbox}

Ввожу путь до файла. У меня это \textbf{data.tsv}. В консоли всякая информация, что файл сохранен в струтуру.

\begin{verbatim}
Размер пути файла: 10
Какой файл открыть: data.tsv
 = = = = = case 0 = = = = =
0
 = = = = = case 1 = = = = =
8
 = = = = = case 2 = = = = =
m74y7d
 = = = = = case 3 = = = = =
5
 = = = = = case 4 = = = = =
Alpha
 = = = = = case 5 = = = = =
9
 = = = = = case 6 = = = = =
Podushkin
 = = = = = case 7 = = = = =
1
 = = = = = case 0 = = = = =
1
 = = = = = case 1 = = = = =
8
 = = = = = case 2 = = = = =
d4rh75
 = = = = = case 3 = = = = =
4
 = = = = = case 4 = = = = =
Ford
 = = = = = case 5 = = = = =
7
 = = = = = case 6 = = = = =
Kamerov
 = = = = = case 7 = = = = =
1
 = = = = = case 0 = = = = =
2
 = = = = = case 1 = = = = =
8
 = = = = = case 2 = = = = =
fruf45
 = = = = = case 3 = = = = =
3
 = = = = = case 4 = = = = =
BMW
 = = = = = case 5 = = = = =
8
 = = = = = case 6 = = = = =
Rozetkov
 = = = = = case 7 = = = = =
0
 = = = = = case 0 = = = = =
3
 = = = = = case 1 = = = = =
8
 = = = = = case 2 = = = = =
f54h5
 = = = = = case 3 = = = = =
5
 = = = = = case 4 = = = = =
Tesla
 = = = = = case 5 = = = = =
7
 = = = = = case 6 = = = = =
Printov
 = = = = = case 7 = = = = =
1
 = = = = = case 0 = = = = =
4
 = = = = = case 1 = = = = =
8
 = = = = = case 2 = = = = =
85hf4ifj
 = = = = = case 3 = = = = =
4
 = = = = = case 4 = = = = =
Lada
 = = = = = case 5 = = = = =
6
 = = = = = case 6 = = = = =
Banker
 = = = = = case 7 = = = = =
0
Нажмите любую клавишу для продолжения...
\end{verbatim}

Нажимаю любую клавишу. В меню выбираю пунтк 3, чтобы вывести таблицу в консоль.

\begin{tcolorbox}
\begin{verbatim}
| ID   | байт Номер    | байт Марка    | байт Фамилия      | Осмотр     |
| ---- | ---- -------- | ---- -------- | ---- ------------ | ---------- |
| 0    | 8    m74y7d   | 5    Alpha    | 9    Podushkin    | Не пройден |
| 1    | 8    d4rh75   | 4    Ford     | 7    Kamerov      | Не пройден |
| 2    | 8    fruf45   | 3    BMW      | 8    Rozetkov     | Пройден    |
| 3    | 8    f54h5    | 5    Tesla    | 7    Printov      | Не пройден |
| 4    | 8    85hf4ifj | 4    Lada     | 6    Banker       | Пройден    |
Нажмите любую клавишу для продолжения...   
\end{verbatim}
\end{tcolorbox}

Структуры успешно обновилась.

\subsubsection{Если файл не найден}

Выбираю из меню пункт 1, чтобы открыть файл. Ввожу размер файла и не существующий путь

\begin{tcolorbox}
\begin{verbatim}
Размер пути файла: 10
Какой файл открыть: uirg
\end{verbatim}
\end{tcolorbox}

Появиться сообщение, что файл не найден. Также будет меню, чтобы выйти в главное меню, либо же остаться открывать файл.

\begin{tcolorbox}
\begin{verbatim}
Файл не найден!

Меню:
1. Открыть файл
0. Выйти в главное меню
\end{verbatim}
\end{tcolorbox}


    % = = = = = Вывод ЛР
    \labconclusion{}

\end{document}
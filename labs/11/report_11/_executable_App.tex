%<<<===
\subsection{Ввод данных}

Запускаю программу

\begin{tcolorbox}
\begin{verbatim}
Меню:
1. Открыть файл     
2. Ввод данных      
3. Вывод данных     
4. Сортирова по полю
5. Удалить запись    
6. Сохранить как     
0. Выйти из программы
\end{verbatim}
\end{tcolorbox}

Выбираю пункт 2

\begin{tcolorbox}
\begin{verbatim}
Размер строки номера: 
\end{verbatim}
\end{tcolorbox}

Ввожу размер строки номера

\begin{tcolorbox}
\begin{verbatim}
Размер строки номера: 8  
Номер машины: 
\end{verbatim}
\end{tcolorbox}

Ввожу сам номер

\begin{tcolorbox}
\begin{verbatim}
Размер строки номера: 8  
Номер машины: fruf45

Размер строки марки машины: 
\end{verbatim}
\end{tcolorbox}

Ввожу размер строки марки 

\begin{tcolorbox}
\begin{verbatim}
Размер строки номера: 8
Номер машины: fruf45

Размер строки марки машины: 3
Марка машины: 
\end{verbatim}
\end{tcolorbox}

Ввожу марку машины

\begin{tcolorbox}
\begin{verbatim}
Размер строки номера: 8
Номер машины: fruf45

Размер строки марки машины: 3
Марка машины: BMW

Размер строки фамилии владельца: 
\end{verbatim}
\end{tcolorbox}

Ввожу размер фамилии

\begin{tcolorbox}
\begin{verbatim}
Размер строки номера: 8
Номер машины: fruf45

Размер строки марки машины: 3
Марка машины: BMW

Размер строки фамилии владельца: 10
Фамилия владельца:
\end{verbatim}
\end{tcolorbox}

Ввожу фамилию владельца

\begin{tcolorbox}
\begin{verbatim}
Размер строки номера: 8
Номер машины: fruf45

Размер строки марки машины: 6
Марка машины: БМВ

Размер строки фамилии владельца: 10
Фамилия владельца: Rozetkov

Осмотр:       
1. Есть осмотр
2. Нет осмотра
\end{verbatim}
\end{tcolorbox}

Ввожу 1. (Если нажать другую клавишу - повторит цикл ввода осмотра)
%===>>>

%<<<===
\subsection{Вывод данных в консоль}

Открылось главное меню. Выбираю 3-ий пункт.

\begin{tcolorbox}
\begin{verbatim}
| ID   | байт Номер    | байт Марка    | байт Фамилия      | Осмотр     |
| ---- | ---- -------- | ---- -------- | ---- ------------ | ---------- |
| 0    | 8    fruf45   | 3    BMW      | 10   Rozetkov     | Пройден    |
Нажмите любую клавишу для продолжения...
\end{verbatim}
\end{tcolorbox}

Жму любую клавишу. Попадаю в главное меню. Выбираю пункт 1. Заполняю данные... Выбираю пункт 2 для просмотра.

\begin{tcolorbox}
\begin{verbatim}
| ID   | байт Номер    | байт Марка    | байт Фамилия      | Осмотр     |
| ---- | ---- -------- | ---- -------- | ---- ------------ | ---------- |
| 0    | 8    fruf45   | 3    BMW      | 10   Rozetkov     | Пройден    |
| 1    | 8    m74y7d   | 10   Alpha    | 10   Podushkin    | Не пройден |
| 2    | 8    d4rh75   | 4    Ford     | 10   Kamerov      | Не пройден |
| 3    | 8    diejd    | 4    Lada     | 10   Zaraydkov    | Не пройден |
Нажмите любую клавишу для продолжения...
\end{verbatim}
\end{tcolorbox}
%===>>>

%<<<===
\subsection{Сортировка по полям}

Нажимаю любую клавишу. Попадаю в главное меню. Выбираю пункт 4.

\begin{tcolorbox}
\begin{verbatim}
1. Сортировка по номеру
2. Сортировка по марке  
3. Сортировка по фамилии
4. Сортировка по осмотру
0. Выйти

По какой строке сортировать: 
\end{verbatim}
\end{tcolorbox}

Выбираю пункт 1. Попадаю в главное меню. Нажимаю пункт 3 для вывода таблицы.

\begin{tcolorbox}
\begin{verbatim}
| ID   | байт Номер    | байт Марка    | байт Фамилия      | Осмотр     |
| ---- | ---- -------- | ---- -------- | ---- ------------ | ---------- |
| 0    | 8    d4rh75   | 4    Ford     | 10   Kamerov      | Не пройден |
| 1    | 8    diejd    | 4    Lada     | 10   Zaraydkov    | Не пройден |
| 2    | 8    fruf45   | 3    BMW      | 10   Rozetkov     | Пройден    |
| 3    | 8    m74y7d   | 10   Alpha    | 10   Podushkin    | Не пройден |
Нажмите любую клавишу для продолжения...
\end{verbatim}
\end{tcolorbox}

Жму любую клавишу. Попадаю в главное меню. Выбираю пункт 4 для сортировки. Выбираю пункт 2. Попадаю в главное меню. Выбираю пункт 3 для вывода таблицы.

\begin{tcolorbox}
\begin{verbatim}
| ID   | байт Номер    | байт Марка    | байт Фамилия      | Осмотр     |
| ---- | ---- -------- | ---- -------- | ---- ------------ | ---------- |
| 0    | 8    m74y7d   | 10   Alpha    | 10   Podushkin    | Не пройден |
| 1    | 8    fruf45   | 3    BMW      | 10   Rozetkov     | Пройден    |
| 2    | 8    d4rh75   | 4    Ford     | 10   Kamerov      | Не пройден |
| 3    | 8    diejd    | 4    Lada     | 10   Zaraydkov    | Не пройден |
Нажмите любую клавишу для продолжения...
\end{verbatim}
\end{tcolorbox}

Жму любую клавишу. Попадаю в главное меню. Выбираю пункт 4 для сортировки. Выбираю пункт 3. Попадаю в главное меню. Выбираю пункт 3 для вывода таблицы.

\begin{tcolorbox}
\begin{verbatim}
| ID   | байт Номер    | байт Марка    | байт Фамилия      | Осмотр     |
| ---- | ---- -------- | ---- -------- | ---- ------------ | ---------- |
| 0    | 8    d4rh75   | 4    Ford     | 10   Kamerov      | Не пройден |
| 1    | 8    m74y7d   | 10   Alpha    | 10   Podushkin    | Не пройден |
| 2    | 8    fruf45   | 3    BMW      | 10   Rozetkov     | Пройден    |
| 3    | 8    diejd    | 4    Lada     | 10   Zaraydkov    | Не пройден |
Нажмите любую клавишу для продолжения...
\end{verbatim}
\end{tcolorbox}

Жму любую клавишу. Попадаю в главное меню. Выбираю пункт 4 для сортировки. Выбираю пункт 4. Попадаю в главное меню. Выбираю пункт 3 для вывода таблицы.

\begin{tcolorbox}
\begin{verbatim}
| ID   | байт Номер    | байт Марка    | байт Фамилия      | Осмотр     |
| ---- | ---- -------- | ---- -------- | ---- ------------ | ---------- |
| 0    | 8    m74y7d   | 10   Alpha    | 10   Podushkin    | Не пройден |
| 1    | 8    d4rh75   | 4    Ford     | 10   Kamerov      | Не пройден |
| 2    | 8    diejd    | 4    Lada     | 10   Zaraydkov    | Не пройден |
| 3    | 8    fruf45   | 3    BMW      | 10   Rozetkov     | Пройден    |
Нажмите любую клавишу для продолжения...
\end{verbatim}
\end{tcolorbox}
%===>>>

%<<<===
\subsection{Удаление поля}

Жму любую клавишу. Попадаю в главное меню. Выбираю пункт 5 для удаления элемента.

\begin{tcolorbox}
\begin{verbatim}
Какой элемент удалить:
\end{verbatim}
\end{tcolorbox}

Жму 2. Жму Enter. Попадаю в главное меню. Выбираю пункт 3 для вывода таблицы.

\begin{tcolorbox}
\begin{verbatim}
| ID   | байт Номер    | байт Марка    | байт Фамилия      | Осмотр     |
| ---- | ---- -------- | ---- -------- | ---- ------------ | ---------- |
| 0    | 8    m74y7d   | 10   Alpha    | 10   Podushkin    | Не пройден |
| 1    | 8    d4rh75   | 4    Ford     | 10   Kamerov      | Не пройден |
| 2    | 8    fruf45   | 3    BMW      | 10   Rozetkov     | Пройден    |
Нажмите любую клавишу для продолжения...
\end{verbatim}
\end{tcolorbox}

Нажимаю любую клавишу. Переходу в главное меню. Выбираю пункт 5. Ввожу 55. Попадаю в главное меню. Выбираю пункт 3 для вывода таблицы.

\begin{tcolorbox}
\begin{verbatim}
| ID   | байт Номер    | байт Марка    | байт Фамилия      | Осмотр     |
| ---- | ---- -------- | ---- -------- | ---- ------------ | ---------- |
| 0    | 8    m74y7d   | 10   Alpha    | 10   Podushkin    | Не пройден |
| 1    | 8    d4rh75   | 4    Ford     | 10   Kamerov      | Не пройден |
| 2    | 8    fruf45   | 3    BMW      | 10   Rozetkov     | Пройден    |
Нажмите любую клавишу для продолжения...
\end{verbatim}
\end{tcolorbox}

Таблица не поменялась, так как нет такого ID.

Нажимаю 0, чтобы выйти из программы.
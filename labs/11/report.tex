\documentclass[12pt, a4paper]{article}

\newcommand{\kafedra}{ИИТ} % Кафедра X

\newcommand{\numberOfLab}{2} % Лабораторная работа №X
\newcommand{\semestr}{2} % За X семестр
\newcommand{\distiplina}{ОАиП} % По дисциплине <<X>>
\newcommand{\labTitle}{Структуры, перечисления, объединения} % Тема: <<X>>

% Выполнил
\newcommand{\kurs}{1} % Студент X курса
\newcommand{\group}{ПО-4 (1)} % группы X
\newcommand{\labAuthor}{Галанин П. И.} % X

% Проверил
\newcommand{\teacherStatus}{ст. преподаватель} % X
\newcommand{\teacher}{Хацкевич М. В.} % X

\newcommand{\labDate}{Брест, 2020} % X

\usepackage{../../sty/encoding} % кодировка
\usepackage{../../sty/titlePage} % титульный лист
\usepackage{../../sty/fields} % поля
\usepackage{../../sty/imgs} % картинки
\usepackage{../../sty/code} % исходный код
\usepackage{../../sty/labData} % для лабораторной

\begin{document}

% титульный лист
\maketitle
\setcounter{page}{2}

% содержание
\renewcommand{\contentsname}{Содержание}
\tableofcontents
\newpage

% контент
\labheading

\labgoal{Изучить синтаксис и правила работы со структурами. Реализовать программу с применением структур, перечислений и объединений.}

\labreport

\section{Условие}

Создать тип структуры согласно варинту, организовать поля этой структуры так, чтобы они содержали объединение, перечисление (можно добавить дополнительные поля) и битовое поле.

Создать массив структур, содержащий информацию согласно варианту индивидуального задания.

Реализовать работу с массивом структур через меню: ввод данных в массив, вывод собержимого массива на экран, сортировка по одному полю, удаления записи по заданному значению поля, выборка записей согласно индивидуального задания.

\begin{center}
    \textbf{Вариант 13}
\end{center}

Сведения об автомобиле состоят из номера, марки, фамилии владельца, признака прохождения техосмотра. Для каждой марки подсчитать количество автомобилей этой марки.

\section{Проект}

\subsection{Структура проекта}

\begin{verbatim}
.
├── Makefile
├── inc
│   ├── clearConsole.h
│   ├── del_data.h
│   ├── encoding.h
│   ├── getch.h
│   ├── input_data.h
│   ├── lab.h
│   ├── main.h
│   ├── menu.h
│   ├── out_data.h
│   ├── pause_console.h
│   └── sort_data
│       ├── sort_data.h
│       ├── sort_data_by_mark_field.h
│       ├── sort_data_by_osmotr_field.h
│       ├── sort_data_by_surname_field.h
│       └── sort_data_in_number_field.h
└── src
    ├── clearConsole.c
    ├── del_data.c
    ├── encoding.c
    ├── getch.c
    ├── input_data.c
    ├── lab.c
    ├── main.c
    ├── menu.c
    ├── out_data.c
    ├── pause_console.c
    └── sort_data
        ├── sort_data.c
        ├── sort_data_by_mark_field.c
        ├── sort_data_by_osmotr_field.c
        ├── sort_data_by_surname_field.c
        └── sort_data_in_number_field.c

4 directories, 31 files
\end{verbatim}

\subsection{Исходный код}

\subsubsection{Makefile}
\lstinputlisting[language=make]{13/Makefile}

\subsubsection{main()}
\lstinputlisting[language=C]{13/inc/main.h}
\lstinputlisting[language=C]{13/src/main.c}

\subsubsection{encoding()}
\lstinputlisting[language=C]{13/inc/encoding.h}
\lstinputlisting[language=C]{13/src/encoding.c}

\subsubsection{lab()}
\lstinputlisting[language=C]{13/inc/lab.h}
\lstinputlisting[language=C]{13/src/lab.c}

\subsubsection{clearConsole()}
\lstinputlisting[language=C]{13/inc/clearConsole.h}
\lstinputlisting[language=C]{13/src/clearConsole.c}

\subsubsection{getch()}
\lstinputlisting[language=C]{13/inc/getch.h}
\lstinputlisting[language=C]{13/src/getch.c}

\subsubsection{pause\_console()}
\lstinputlisting[language=C]{13/inc/pause_console.h}
\lstinputlisting[language=C]{13/src/pause_console.c}

\subsubsection{menu()}
\lstinputlisting[language=C]{13/inc/menu.h}
\lstinputlisting[language=C]{13/src/menu.c}

\subsubsection{input\_data()}
\lstinputlisting[language=C]{13/inc/input_data.h}
\lstinputlisting[language=C]{13/src/input_data.c}

\subsubsection{out\_data()}
\lstinputlisting[language=C]{13/inc/out_data.h}
\lstinputlisting[language=C]{13/src/out_data.c}

\subsubsection{sort\_data/sort\_data()}
\lstinputlisting[language=C]{13/inc/sort_data/sort_data.h}
\lstinputlisting[language=C]{13/src/sort_data/sort_data.c}

\subsubsection{sort\_data/sort\_data\_in\_number\_field()}
\lstinputlisting[language=C]{13/inc/sort_data/sort_data_in_number_field.h}
\lstinputlisting[language=C]{13/src/sort_data/sort_data_in_number_field.c}

\subsubsection{sort\_data/sort\_data\_by\_mark\_field()}
\lstinputlisting[language=C]{13/inc/sort_data/sort_data_by_mark_field.h}
\lstinputlisting[language=C]{13/src/sort_data/sort_data_by_mark_field.c}

\subsubsection{sort\_data/sort\_data\_by\_surname\_field()}
\lstinputlisting[language=C]{13/inc/sort_data/sort_data_by_surname_field.h}
\lstinputlisting[language=C]{13/src/sort_data/sort_data_by_surname_field.c}

\subsubsection{sort\_data/sort\_data\_by\_osmotr\_field()}
\lstinputlisting[language=C]{13/inc/sort_data/sort_data_by_osmotr_field.h}
\lstinputlisting[language=C]{13/src/sort_data/sort_data_by_osmotr_field.c}

\subsubsection{del\_data()}
\lstinputlisting[language=C]{13/inc/del_data.h}
\lstinputlisting[language=C]{13/src/del_data.c}


\subsection{Исполняемая программа}

\labconclusion{}

\newpage

\section{Приложения}

\end{document}